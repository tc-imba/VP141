\documentclass{article}
\usepackage{enumerate}
\usepackage{amsmath}
\usepackage{amssymb}
\usepackage{graphicx}
\usepackage{subfigure}
\usepackage{geometry}
\usepackage{color}
\usepackage{bm}
\usepackage{indentfirst}
\usepackage{multirow}

\begin{document}

\vspace*{0.25cm}

\hrulefill

\thispagestyle{empty}

\begin{center}
\begin{large}
\sc{UM--SJTU Joint Institute \vspace{0.3em} \\ Physics Laboratory \\(Vp141)}
\end{large}

\hrulefill

\vspace*{5cm}
\begin{Large}
\sc{{Laboratory Report}}
\end{Large}

\vspace{2em}

\begin{large}
\sc{{Exercise 1
\vspace{0.5em}

Measurements of the Moment of Inertia
}}
\end{large}
\end{center}


\vfill

\begin{table}[h!]
\flushleft
\begin{tabular}{lll}
Name: Yihao Liu \hspace*{2em}&
ID: 515370910207\hspace*{2em}\\
Name: Guangzheng Wu \hspace*{2em}&
ID: 515370910175\hspace*{2em}
& Group: 7\\


\\

Date: 21 June 2016 

\end{tabular}
\end{table}

\hfill
\begin{tiny}
[rev. 1.0]
\end{tiny}
\newpage

\section{Introduction}

The objectives of this exercise are to get familiar with the constant-torque method for
measuring the moment of inertia of a rigid body and study the dependence of the moment
of inertia on the change of mass, mass distribution or axis of rotation. Moreover, in this
exercise, the parallel axis theorem (Steiner’s) theorem will be verified.
As a part of measurement technique skills, Learn to measure the time using the
counter-type electronic timer.
\\

Moment of inertia is a property of a rigid body that defines its resistance (inertia) to
a change of angular velocity of rotation about an axis. This qualitative characteristics of
an extended body constrained to rotate about an axis is determined by a combination
of mass and its distribution. The moment of inertia of a rigid bodywith respect to a
certain rotation axis can be calculated mathematically. However, if the body has rela-
tively irregular shape or non-uniformly distributed mass, the calculation may be difficult.
Experimental methods may be used in such cases.

\subsection{Second Law of Dynamics for Rotational Motion}

According to the second law of dynamics for rotational motion about a fixed axis

\begin{equation}\label{eq-1}
M=I\beta
\end{equation}

relates the component of the torque $M$ about the axis of rotation with the moment of
inertia about this axis, and angular acceleration component $\beta$. Therefore, the moment
of inertia $I$ can be found once the torque and the resultant angular acceleration are
measured.\\

Moment of inertia is an additive quantity, $i.e$. if the moment of inertia of rigid body
$A$ about an axis is $I_A$ and the moment of inertia of rigid body $B$ about the same axis is
$I_B$ , then the moment of inertia of the combined rigid body $AB$ composed of $A$ and $B$ is

$$I_{AB}=I_A+I_B$$

\subsection{Parallel Axis Theorem}

If the moment of inertia of a rigid body with mass m about an axis through the center
of mass is $I_0$ . Then for any axis parallel to that axis, the moment of inertia is

\begin{equation}\label{eq-2}
I=I_0+md^2
\end{equation}

where $d$ is the distance between the axes. This result is known as the parallel axis theorem
or Steiner’s theorem.

\subsection{Apparatus and Measurement Method}

The experimental setup is shown in Figure \ref{fig-1}. It consists of a turntable with an
attached photo gate system used for time measurements.\\

Let us assume that, the empty turntable is initially rotating and its moment of inertia
with respect to the rotation axis is $I_1$ . Since the bearings are not frictionless, there
will be a non-zero frictional torque $M_\mu$ causing the turntable to decelerate with angular
acceleration $\beta_1$

\begin{equation}\label{eq-3}
M_\mu=-I_1\beta_1
\end{equation}

On the other hand, assume that a light and inextensible string is wound on a cone
pulley with radius $R$ placed on the axis of rotation. Attached to the other end of the string
passing through a disk pulley, there is a mass $m$ hanging. After the mass is released the
system moves (rotates) with a constant acceleration (angular acceleration) because of a
constant net force (torque). If the mass moves downward with acceleration $a$, the tension
in the string is $T = m(g − a)$. And if the turntable rotates with angular acceleration $\beta_2$ at
the same time, we have $a = R_{\beta_2}$ (we assume that the string does not slip on the pulleys).
The torque the string exerts on the turntable is then $T_R = m(g − R_{\beta_2} )R$, and hence the
equation of motion of the turntable reads

\begin{equation}\label{eq-4}
m(g − R_{\beta_2} )R-M_\mu=I_1\beta_1
\end{equation}

Eliminating $M_\mu$ from Eqs. \ref{eq-3} and \ref{eq-4}, one obtains

\begin{equation}\label{eq-5}
I_1=\frac{mR(g-R\beta_2)}{\beta_2-\beta_1}
\end{equation}

Similarly, if a rigid body of unknown moment of inertia is placed on the turntable, we
may find

\begin{equation}\label{eq-6}
I_2=\frac{mR(g-R\beta_4)}{\beta_4-\beta_3}
\end{equation}

where $\beta_3$ is the magnitude of angular deceleration of the turntable with the body, and $\beta_3$
is its angular acceleration, when the mass $m$ is released.\\

Using the fact that the moment of inertia is an additive quantity, the moment of inertia
of the rigid object placed on the turntable may be found as the difference

\begin{equation}\label{eq-7}
I_3=I_2-I_1
\end{equation}

\subsection{Measurement of the angular acceleration}

In the experiment, two shielding pins are fixed at the edge of the turntable generating
signals at the photo gate with the phase interval of $\pi$ as the turntable rotates. A counter-
type electronic timer is used to measure the number k and the time $t$ of the photo gate
signal. If ($k_m$ , $t_m$ ) and ($k_n$ , $t_n$ ) are two sets of data from the measurement, the angular
displacements are

\begin{equation}\label{eq-8}
\theta_m=k_m\pi=\omega_0t_m+\frac{1}{2}\beta t_m^2
\end{equation}

\begin{equation}\label{eq-9}
\theta_n=k_n\pi=\omega_0t_n+\frac{1}{2}\beta t_n^2
\end{equation}

where $\omega_0$ is the initial angular speed. After eliminating $\omega_0$ in Eq. \ref{eq-8} and \ref{eq-9}, the angular
acceleration is obtained as

\begin{equation}\label{eq-10}
\beta=\frac{2\pi(k_nt_m-k_mt_n)}{t_n^2t_m-t_m^2t_n}
\end{equation}

\section{Measurement}
\begin{enumerate}
\item
Measure the mass of the weight, hoop, disk, and cylinder, and the radius of the cone
pulley and the cylinder (as requested by the instructor). Calculate the moment of
inertia of the hoop and the disk mathematically. Find the local gravity acceleration
on Internet.
\item
Turn on the electronic timer and switch to mode 1-2 (single gate, multiple pulses).
\item
Place the instrument close to the edge of the desk and stretch the pulley arm outside.
\item
Level the turntable with the bubble level.
\item
Make the turntable rotating and press the start button of the timer. After at least
8 signals are recorded, stop the turntable and write down the data.
\item
Attach the weight to one end of the string. Place the string on the disk pulley,
thread through the hole on the arm, and wind the string around the 3rd winder of
the cone pulley. Adjust the arm holder so that the string is crossing the center of
the hole.
\item
Release the weight and start the timer. Stop the turntable when the weight hits the
floor. Write down the recorded data.
\item
The moment of inertia of the empty turntable is obtained by using the formulae in
Section 3 and the data from step 5 and 7. Repeat the step 5 ∼ 7 with a rigid object
placed on the turntable. Eq. \ref{eq-7} is used to find the moment of inertia of the rigid
object.

\end{enumerate}

In the experiment, each student should measure the moment of inertia of the empty
turntable, the hoop, the disk and the cylinder, as well as verify the parallel-axis theorem
by placing two cylinders off the axis while keeping their center of mass on axis in order
to keep the rotation steady.\\

The distance of the holes to the center of the turntable are about 45, 60, 75, 90, 105
mm respectively. The timer resolution is 0.0001 s, and the error is 0.004\%.



\input{results.tex}

\section{Measurement uncertainty analysis}

\begin{align*}
u_\beta&=\sqrt{
\left(\frac{\partial \beta}{\partial t_n}\right)^2u_{tn}^2+
\left(\frac{\partial \beta}{\partial t_m}\right)^2u_{tm}^2
}\\
&=2\pi\sqrt{
\left[\frac{k_mt_n^2t_m-2k_nt_nt_m^2+k_nt_m^3}{(t_n^2t_m-t_m^2t_n)^2}\right]^2u_{tn}^2+
\left[\frac{k_nt_m^2t_n-2k_mt_mt_n^2+k_mt_n^3}{(t_n^2t_m-t_m^2t_n)^2}\right]^2u_{tm}^2
}
\end{align*}

\begin{align*}
u_I&=\sqrt{
\left(\frac{\partial I}{\partial m}\right)^2u_{m}^2+
\left(\frac{\partial I}{\partial R}\right)^2u_{R}^2+
\left(\frac{\partial I}{\partial \beta_1}\right)^2u_{\beta_1}^2+
\left(\frac{\partial I}{\partial \beta_2}\right)^2u_{\beta_2}^2
}\\
&=\left\lbrace
\left[\frac{R(g-R\beta_2)}{\beta_2-\beta_1}\right]^2u_{m}^2+
\left(\frac{mg-2m\beta_2R}{\beta_2-\beta_1}\right)^2u_{R}^2+
\right.\\&\quad\left.
\left[\frac{mR(g-R\beta_2)}{(\beta_2-\beta_1)^2}\right]^2u_{\beta_1}^2+
\left[\frac{-mR^2(\beta_2-\beta_1)-mR(g-R\beta_2)}{(\beta_2-\beta_1)^2}\right]^2u_{\beta_2}^2
\right\rbrace^{0.5}
\end{align*}
$$\beta=\frac{2\pi(k_nt_m-k_mt_n)}{t_n^2t_m-t_m^2t_n}$$
$$I_1=\frac{mR(g-R\beta_2)}{\beta_2-\beta_1}$$




\end{document}